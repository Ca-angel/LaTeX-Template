\documentclass[11pt, a4paper]{book}

\usepackage[utf8]{inputenc}
\usepackage[spanish]{babel}
\usepackage{graphicx}
\usepackage[left=2cm, right=2cm, top=2cm, bottom=2cm]{geometry}
\usepackage{multirow}
\usepackage{xcolor}
\usepackage{float}
\usepackage{caption}
\usepackage{subcaption}
\definecolor{darktan}{rgb}{0.57, 0.51, 0.32}
\usepackage{lipsum}


\usepackage{hyperref}
\usepackage{bookmark}
\bookmarksetup{
  numbered,
  open
}
\renewcommand*{\thesection}{\arabic{section}}


\author{Carlos Alberto Conde Angel}
\title{}

\begin{document}
\include{caratulat}
\tableofcontents


\chapter*{Sensores}
\addcontentsline{toc}{chapter}{Sensores} \markboth{Sensores}{}
	
\section{Variables y magnitudes físicas.}




\begin{thebibliography}{X}
\addcontentsline{toc}{chapter}{Bibliografía} \markboth{Bibliografía}{}

\bibitem{Dan} \textsc{L. G. Corona Ramírez, G. S. Abarca Jiménez y J. Mares Carreño}, \textit{Sensores y actuadores. Aplicaciones con arduino}. Azcapotzalco: GRUPO EDITORIAL PATRIA, S.A. DE C.V., 2014.
\bibitem{Baz} \textsc{ Todo sobre circuitos.} \textit{"Relé: qué es y cómo funciona"}. https://www.circuitos-electricos.com/rele-que-es-y-como-funciona/ (accedido el 2 de octubre de 2022).


\end{thebibliography}
















\end{document}